\documentclass{beamer}
\mode<presentation>
{
  \usetheme{default} 
  \usecolortheme{default}
  \usefonttheme{default}
  \setbeamertemplate{navigation symbols}{}
  \setbeamertemplate{caption}[numbered]
} 

\usepackage[portuguese]{babel}
\usepackage[utf8]{inputenc}

\title[]{Pesquisa de Algoritmo Estocástico para o Problema U-Curve}
\author{Gustavo Estrela}
\institute{Universidade de São Paulo e Texas A\&M University}
\date{\today}

\begin{document}

\begin{frame}
  \titlepage
\end{frame}

% Uncomment these lines for an automatically generated outline.
%\begin{frame}{Outline}
%  \tableofcontents
%\end{frame}

\section{Objetivo}
\begin{frame}{Objetivo}
\begin{itemize}
  \item Estudo do algoritmo IUBB.
      \vskip 5em
  \item Implementação de uma variante estocástica do IUBB.
\end{itemize}
\end{frame}

\section{O Algoritmo IUBB}
\begin{frame}{O Algoritmo IUBB}
    \begin{itemize}
        \item{Baseado no UBB.}
        \vskip 5em
        \item{Duas ideias principais:}
        \begin{itemize}
            \vskip 1em
            \item{Atualização iterativa de cadeias ótimas.}
            \vskip 1em
            \item{Uso da bisecção na procura do mínimo de uma cadeia.}
        \end{itemize}
    \end{itemize}
\end{frame}

\section{Introdução do erro}
\begin{frame}{Introdução do erro}
    \begin{itemize}
        \item{Assumimos erro com distribuição normal com média zero e
            variância $\sigma^2$.}
        \item{Efeitos do erro no algoritmo IUBB.}
    \end{itemize}
\end{frame}

\end{document}

